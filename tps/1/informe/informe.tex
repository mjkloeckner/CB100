\documentclass[12pt]{article}
\usepackage[spanish]{babel}
\usepackage{natbib}
\usepackage{url}
\usepackage[utf8]{inputenc}
\usepackage{amsmath}
\usepackage{graphicx}
\usepackage{parskip}
\usepackage{fancyhdr}
\usepackage{vmargin}
\usepackage[ddmmyy]{datetime}
\usepackage{anyfontsize}
\usepackage{helvet}
\renewcommand{\familydefault}{phv}
\usepackage{xcolor}
\usepackage[T1]{fontenc}

% \setmarginsrb{leftmargin}{topmargin}{rightmargin}{bottommargin}%
%          {headheight}{headsep}{footheight}{footskip}
\setmarginsrb{2.5 cm}{2.25 cm}{2.5 cm}{1.50 cm}
           {1 cm}{1.25 cm}{1 cm}{1.50 cm}

\usepackage{color}
\usepackage{hyperref}
\hypersetup{
    colorlinks=true, % set true if you want colored links
    linktoc=all,     % set to all if you want sections and subsections linked
    linkcolor=blue,  % choose some color if you want links to stand out
    urlcolor=blue,
}

\usepackage{subfig}
\usepackage[labelformat=empty]{caption}

\usepackage{fancyvrb}
\fvset{xleftmargin=\mathindent}

% code blocks
\usepackage{verbatimbox}
\newenvironment{fullgrayverb}
{\verbbox}
{\endverbbox\par\colorbox{gray!25}{\parbox{\textwidth}{\theverbbox}}\par}

% inline code blocks
\usepackage{tcolorbox}
\newcommand\mystrut{\rule[-3pt]{0pt}{12pt}}
\newtcbox{\code}{on line, boxrule=0pt, boxsep=0pt, top=0pt,
left=0pt, bottom=0pt, right=0pt, colback=gray!25, colframe=white,
fontupper={\ttfamily\mystrut}}

%%%%%%%%%%%%%%%%%%%%%%%%%%%%%%%%%%%%%%%%%%%%%%%%%%%%%%%%%%%%%%%%%%%%%%%%%%%%%%%%
% DATOS GENERALES

\title{Números Primos}          % Titulo del trabajo.
\author{Martin J. Klöckner}     % Nombre y apellido
\newcommand{\padron}{123456}    % Padrón
\newcommand{\tpnumber}{1}       % Número de trabajo práctico
\date{\today}                   % Fecha (automática)

\makeatletter
\let\thetitle\@title
\let\theauthor\@author
\let\thedate\@date
\makeatother

\pagestyle{fancy}
\fancyhf{}
\rhead{\theauthor}
\lhead{\thetitle}
\cfoot{\thepage}

%%%%%%%%%%%%%%%%%%%%%%%%%%%%%%%%%%%%%%%%%%%%%%%%%%%%%%%%%%%%%%%%%%%%%%%%%%%%%%%%
\begin{document}

\begin{titlepage}
    \vspace*{-2.5cm}
    {\centering
    \includegraphics[width=1.00\textwidth]{img/logofiuba.png}\\[2.25 cm]}
    \centering
    \textsc{\Large CB100}\\[0.2 cm]
    \textsc{\large Algoritmos y Estructuras de Datos}\\[4 cm]
    \textcolor{cyan}{{\fontsize{40}{60}\selectfont \bfseries \thetitle}}\\[0.5cm]
    {\Large \bfseries Trabajo Práctico N$^\circ$\tpnumber}\\[5cm]


    \vfill
    % \vspace{10em}
    \noindent\makebox[\linewidth]{\rule{\textwidth}{0.4pt}}\\[0.5cm]
    \begin{minipage}{.46\textwidth}
    \textbf{Autor}\\
    \theauthor
    \end{minipage}%
    \begin{minipage}{.34\textwidth}
    \textbf{Legajo}\\
    \padron
    \end{minipage}%
    \begin{minipage}{.2\textwidth}
     \begin{flushright}
        \textbf{Fecha}\\
        \thedate
    \end{flushright}
    \end{minipage}
\end{titlepage}

%%%%%%%%%%%%%%%%%%%%%%%%%%%%%%%%%%%%%%%%%%%%%%%%%%%%%%%%%%%%%%%%%%%%%%%%%%%%%%%%

{
    \hypersetup{linkcolor=black} % colorlinks=true option is used
    \tableofcontents
    \pagebreak
}

%%%%%%%%%%%%%%%%%%%%%%%%%%%%%%%%%%%%%%%%%%%%%%%%%%%%%%%%%%%%%%%%%%%%%%%%%%%%%%%%

\section{Introducción}

En este trabajo práctico se desarrolla una aplicación de consola que busca
números primos hasta un valor máximo determinado por el usuario, cuando
finaliza, los imprime en un archivo de texto plano en el directorio de ejecución
de la aplicación.

El lenguaje de programacion utilizado para el desarrollo de la aplicación es C++
y el algoritmo utilizado para hallar los números primos es la Criba de
Eratóstenes.

\subsection{Números Primos}

Los números primos por definición son aquellos números positivos que solo son
divisibles por \code{1} y por si mismos, divisibles en términos de que el resto
de la división es nulo. El \code{0} y el \code{1} son dos casos particulares en
los cuales ambos son considerados no primos, el \code{0} por razones obvias, no
está definida la división por \code{0}, y el \code{1} por que solo es divisible
por si mismo, lo cual hace que solo sea divisible por un número.

El número \code{7}, por ejemplo, es un número primo, ya que solo es divisible
por \code{1} y por \code{7}. El número \code{8} no es un número primo, ya que
ademas de ser divisible por si mismo y por \code{1}, es divisible por \code{2} y
por \code{4}.

% \subsection{Algoritmo Primitivo para Descartar Números Primos}
% 
% El primer algoritmo que surge casi instantáneamente es un algoritmo en el que se
% itera sobre una lista de números y por cada numero se descartan todos sus
% múltiplos. Este algoritmo cumple la función de descartar números no primos, pero
% no es para nada eficiente.

\subsection{Criba de Eratóstenes}

La criba de Eratóstenes es un algoritmo sofisticado y eficiente para descartar
números no primos de una lista de números. En la Criba de Eratóstenes se itera
número a número sobre una lista ordenada y finita de números naturales,
comenzando por el primero, si se trata de un número primo, se buscan todos los
múltiplos en la lista de números y se descartan, luego se avanza al siguiente
número que no haya sido descartado, el cual siempre resulta un número primo, y
se descartan sus múltiplos, así sucesivamente hasta el final de la lista. 

La criba de Eratóstenes se puede optimizar basándose en la propiedad de que los
números no primos (números compuestos), se pueden expresar como el producto de
números primos, por ejemplo \code{4 = 2*2} o \code{21 = 3*7}, por lo tanto, si
un número no es primo, debe tener al menos un factor primo que sea menor o igual
que su raíz cuadrada, ya que de lo contrario, el producto de los factores
resultaría en un numero mayor.

De la propiedad anterior se deduce que se puede detener la iteración en la lista
de números cuando se llega a la raíz cuadrado del numero máximo en la lista, ya
que los números no primos por encima de éste tendrán algún factor menor que la
raíz cuadrada del numero máximo, de este modo se reduce drásticamente el numero
de iteraciones necesarias para descartar los números no primos de la lista.

\section{Desarrollo}

Para la implementación de la aplicación, como bien se mencionó en la
introducción, se utilizó puramente el lenguaje C++, en particular en su versión
estándar C98. Para organizar el proceso de compilación se utilizó la herramienta
\code{make}.

\subsection{Implementación en C++}

El código fuente de la aplicación esta contenido en un solo archivo:
\code{primos.cpp}.

En la primera parte del archivo se definen dos constantes globales
\code{OUTPUT\_FILE\_PATH}, que representa el nombre del archivo de salida, y
\code{MAXIMO} que representa el largo de la lista de números a analizar.

\begin{fullgrayverb}[\mbox{}]
#define OUTPUT_FILE_PATH "primos.txt"
const unsigned int MAXIMO = 100000000;
\end{fullgrayverb}

Luego se declara la función \code{vectorDiscardNonPrimes}, esta función asigna
el valor \code{false} a los casilleros del vector, que recibe como argumento, en
los cuales el índice es un numero no primo.

\begin{fullgrayverb}[\mbox{}]
// Se declara la función `vectorDiscardNonPrimes`
void vectorDiscardNonPrimes(std::vector<bool>& v);
\end{fullgrayverb}

Se declara la función \code{vectorExportToFilePath} la cual recibe como
argumento un vector, un archivo abierto y una referencia a una variable. Esta
función imprime sobre el archivo las posiciones del vector en donde el
contenido es \code{true}, la cantidad de números impresos los asigna a la
variable \code{primesWritten}.

\begin{fullgrayverb}[\mbox{}]
// Se declara la función `vectorExportToFilePath`
void vectorExportToFilePath(
         const std::vector<bool> v,
         std::ofstream& fp,
         unsigned int &primesWritten);
\end{fullgrayverb}

\subsubsection{Función \code{main}}

La Criba de Eratóstenes opera sobre una lista de números naturales ordenada,
para representarla se utiliza la librería estándar \code{vector}, de la cual
se define un vector de tipo \code{bool} inicialmente con valores \code{true}
indicando que son todos números primos.

\begin{fullgrayverb}[\mbox{}]
// Se define un vector de tipo `bool` de largo `MAXIMO`
// con valores iniciales `true`
std::vector<bool> numeros(MAXIMO, true);
\end{fullgrayverb}

\pagebreak
Para la manipulación del archivo de salida, se utiliza la clase \code{ofstream}
de la librería estándar \code{fstream}.

\begin{fullgrayverb}[\mbox{}]
// Se define el objeto fp para representar el archivo de salida
std::ofstream fp;
\end{fullgrayverb}

Además se define la variable \code{primesFound} la cual luego contendrá la
cantidad de números primos hallados.

\begin{fullgrayverb}[\mbox{}]
unsigned int primesFound;
\end{fullgrayverb}

Para descartar los números no primos del vector \code{numeros} se invoca a la
función \linebreak\code{vectorDiscardNonPrimes}.

\begin{fullgrayverb}[\mbox{}]
vectorDiscardNonPrimes(numeros);
\end{fullgrayverb}

Luego de finalizada la función, se imprime en el archivo de salida los índices
del vector correspondientes a los casilleros que permanecen con un valor
\code{true}, ya que éstos representan los números primos.

Utilizando los métodos \code{open} e \code{is\_open} del objeto \code{fp} se
abre el archivo \code{primos.txt} y se comprueba si hubo algún error, en caso
afirmativo, se informa al usuario haciendo uso de métodos de la librería
estándar \code{iostream} y se termina la ejecución inmediatamente con un código
de error \code{-1}.

\begin{fullgrayverb}[\mbox{}]
fp.open(OUTPUT_FILE_PATH);
if (fp.is_open() == false) {
    std::cerr << "ERROR: No se pudo abrir `" OUTPUT_FILE_PATH "`\n";
    return -1;
}
\end{fullgrayverb}

Para exportar los números primos del vector se invoca a la función
\linebreak\code{vectorExportToFilePath}

\begin{fullgrayverb}[\mbox{}]
vectorExportToFilePath(numeros, fp, primesFound);
\end{fullgrayverb}

Luego de finalizada la impresión en el archivo, se cierra utilizando el método
\code{close}.

\begin{fullgrayverb}[\mbox{}]
fp.close();
\end{fullgrayverb}

Finalmente se imprime un mensaje al usuario indicando que terminó la impresión,
y se informa la cantidad de números primos hallados, esto se hace utilizando
métodos de la librería estándar \code{iostream}.

\begin{fullgrayverb}[\mbox{}]
std::cout << "Se encontraron `" << primesFound << "` números primos\n";
\end{fullgrayverb}

Por ultimo se termina la ejecución con un código \code{0} indicando que el
programa se ejecutó exitosamente.

\begin{fullgrayverb}[\mbox{}]
return 0;
\end{fullgrayverb}

\subsubsection{Función \code{vectorDiscardNonPrimes}}

La función \code{vectorDiscardNonPrimes} recibe una referencia a un vector de
tipo \code{bool} como argumento y asigna \code{false} a los casilleros en los
cuales el índice del mismo es un número no primo. 

La función comienza descartando los casos particulares, \code{0} y \code{1}, por
lo que asigna \code{false} a los índices respectivos directamente.

\begin{fullgrayverb}[\mbox{}]
// 0 y 1 no son primos
numeros[0] = numeros[1] = false;
\end{fullgrayverb}

Luego itera sobre el vector asignando \code{false} a todos los múltiplos de los
números primos, hasta llegar a la raíz cuadrada del numero máximo de la lista, o
lo que es equivalente, hasta que el indice al cuadrado sea menor al numero
máximo.
% Para determinar la raíz cuadrada del numero máximo se utiliza la función
% \code{sqrt} de la librería estándar \code{cmath} 

\begin{fullgrayverb}[\mbox{}]
for (size_t i = 2; i*i < MAXIMO; ++i) {
    if(vector[i] == true) {
        for (size_t j = i; j <= (MAXIMO/i); ++j) {
            vector[i*j] = false;
        }
    }
}
\end{fullgrayverb}

\subsubsection{Función \code{vectorExportToFilePath}}

La función \code{vectorExportToFilePath} recibe como argumento un vector, una
archivo abierto y una referencia a una variable. Luego de invocada la función
imprime sobre el archivo los índices del vector en los cuales el contenido es
\code{true}, a la variable que recibe le asigna la cantidad de índices impresos.

Al comienzo de la función se asigna a la variable \code{primesWritten} cero.

\begin{fullgrayverb}[\mbox{}]
primesWritten = 0;
\end{fullgrayverb}

Para la impresión en el archivo, la función recorre el vector en búsqueda de
aquellos casilleros que contengan \code{true}, en los casos afirmativos imprime
el índice del vector al archivo e incrementa el contenido de la variable
\code{primesWritten}, los casilleros que contienen \code{false} son ignorados.

\begin{fullgrayverb}[\mbox{}]
for (size_t i = 2; i < v.size(); ++i) {
    if(v[i]) {
        fp << i << std::endl;
        primesWritten++;
    }
}
\end{fullgrayverb}

\subsection{Makefile}

% Si bien es importante la utilización de buenas practicas en el código fuente de
% la aplicación, muchas veces el proceso de compilación no se tiene en cuenta,
% pero es también muy importante para generar una aplicación robusta y fácil de
% diagnosticar en caso de fallas.

Para organizar el proceso de compilación de la aplicación se utiliza la
herramienta \code{GNU make}, la cual lee un archivo de nombre exclusivo
\code{Makefile} que posee su propia sintaxis.

El archivo \code{Makefile} utilizado intenta ser reutilizable y de fácil
modificación, es por esto que al comienzo del mismo se utilizan variables para
almacenar las principales opciones que el usuario podría querer modificar, por
ejemplo el nombre del compilador utilizado, el nombre final del archivo, entre
otras.

\begin{fullgrayverb}[\mbox{}]
CC := g++
CFLAGS := -Wall -Wshadow -pedantic -ansi -std=c++98 -O3
SRCS := $(wildcard *.cpp)
OBJS := $(SRCS:.cpp=.o)
TARGET := primos
\end{fullgrayverb}

Para compilar el programa y enlazar cabeceras, en caso de que haya, se utilizan
las siguientes reglas:

\begin{fullgrayverb}[\mbox{}]
$(TARGET): $(OBJS)
    $(CC) $(CFLAGS) -o $@ $^

%.o: %.cpp
    $(CC) $(CFLAGS) -c $< -o $@
\end{fullgrayverb}

La regla que contiene \code{\$(TARGET)} se expande al nombre final de la
aplicación, luego de los dos puntos se indican las dependencias, las cuales se
expanden a los archivos objeto (los que terminan en \code{.o})

Para compilar los archivos objetos se utiliza la segunda regla, la cual
corresponde a una \code{pattern rule}, una extensión exclusiva de \code{GNU}, 
esta \code{pattern rule} compila todos los archivos terminados en \code{.cpp} a
archivos objeto terminados en \code{.o}.

\subsection{Optimizaciones del Compilador}

La mayoría de los compiladores de C++ modernos proporcionan una opción para
activar optimizaciones. Al utilizar estas optimizaciones el compilador
intenta eliminar código redundante y optimizar mediante técnicas avanzadas el
archivo binario final, de este modo se logra reducir el uso de memoria, el
tiempo de ejecución y por consiguiente el consumo de energía, entre otras cosas.

En el archivo \code{Makefile} de la aplicación por defecto se utiliza el máximo
de optimizaciones posibles \code{-O3}, de ésta manera se obtiene un archivo
binario superior en rendimiento que el que se obtendría con las optimizaciones
desactivadas.

% Teniendo un compilador que proporcione optimizaciones del código resulta
% conveniente activarla ya que reduce el tiempo de ejecución hasta en un
% \code{406\%}.

\subsubsection{Tiempo de Ejecución}

En el siguiente bloque se muestra la salida del programa utilizando el máximo de
optimizaciones posible del compilador \code{-O3}, se puede apreciar en la salida
del programa el tiempo de ejecución del mismo.

\begin{fullgrayverb}
Se encontraron `5761455` números primos en `11.36` segundos
\end{fullgrayverb}

Desactivando las optimizaciones del compilador, compilando y ejecutando
nuevamente la aplicación, se obtiene el siguiente mensaje en la consola:

\begin{fullgrayverb}
Se encontraron `5761455` números primos en `46.21` segundos
\end{fullgrayverb}

Comparando los dos resultados, se puede observar que al utilizar las
optimizaciones de compilador el tiempo de ejecución se reduce en
aproximadamente \code{75\%}.

% Como se puede observar en el segundo caso, el tiempo de ejecución se incrementa
% en aproximadamente \code{306\%}, dicho de otro modo, utilizando las
% optimizaciones el tiempo de ejecucion se reduce en aproximadamente \code{75\%}.

\section{Instalación}

Para instalar la aplicación usted debe de disponer de una copia del código
fuente, si no posee una, puede obtenerla ingresando al
\href{https://github.com/mjkloeckner/CB100}{Repositorio de Github}, de allí
podrá descargar una copia, o bien puede clonar el repositorio utilizando
\code{git} con el siguiente comando:

\begin{fullgrayverb}
$ git clone https://github.com/mjkloeckner/CB100.git
\end{fullgrayverb}$

Tenga en cuenta que si clona el Repositorio, o lo descarga del repositorio de
Github, tendrá que navegar al directorio \code{tps/1}

\subsection{Sistemas basados en UNIX}

Compruebe que este en el mismo directorio que el archivo \code{primos.cpp}. Luego,
compile el código con el programa \code{make}

\begin{fullgrayverb}
$ make
\end{fullgrayverb}$

Luego puede ejecutar la aplicación de la siguiente manera:

\begin{fullgrayverb}
$ ./primos
\end{fullgrayverb}$

\pagebreak
\subsection{Windows}

En el caso de utilizar Windows como sistema operativo, usted puede configurar
\href{https://en.wikipedia.org/wiki/Windows_Subsystem_for_Linux}{WSL}, el cual
virtualiza un sistema basado en UNIX, para hacerlo puede seguir la
\href{https://learn.microsoft.com/en-us/windows/wsl/install}{guia oficial de
Microsoft}. Una vez configurado WSL siga los pasos para los sistemas basados en
UNIX

De manera análoga si usted dispone de un compilador de C++ puede compilar la
aplicación directamente desde la consola, sin la necesidad de utilizar la
herramienta \code{make}, para eso ejecute el siguiente comando:

% Define your variable
\newcommand{\myvariable}{\textbf{Hello}}

\begin{fullgrayverb}
$ g++ -Wall -Wshadow -ansi -std=c++98 -O3 primos.cpp -o primos
\end{fullgrayverb}$


Tenga en cuenta la sintaxis de su compilador ya que puede variar, el comando
anterior está previsto para \href{https://www.mingw-w64.org/}{MinGW-w64}

Luego de compilado la aplicación usted la puede ejecutar de la siguiente manera 

\begin{fullgrayverb}
$ ./primos
\end{fullgrayverb}$

\section{Conclusión}

Luego de finalizar el desarrollo de la aplicación, se puede concluir que la
combinación de C++ con \code{GNU Make}, resulta muy versátil, y en este caso en
un alto rendimiento.

El lenguaje C++ es uno de los lenguajes de mayor rendimiento en la lista de
lenguajes de programación de alto nivel actuales, junto con C, aunque a
diferencia de éste, C++ ofrece una amplia variedad de librerías estándar, lo que
mejora su robustez y permite al usuario abstraerse de ciertas implementaciones
que resultan irrelevantes en ciertos casos, por ejemplo, en el desarrollo de
esta aplicación, el objecto \code{vector} utilizado para representar la lista de
números naturales. De no existir la librería estándar \code{vector} uno tendría
que ensuciarse las manos e implementar un tipo de dato para representar los
números naturales, ya que no alcanza con los tipos de datos primitivos ofrecidos
por el lenguaje.

Ademas de ahorrar tiempo al programador, las librerías estándar ofrecen mayor
seguridad que si se tratase de un tipo de dato creado por el usuario, ya que las
principales están respaldadas por organizaciones y grupos de estándares como
ISO, IEC, entre otros.

\pagebreak
\section{Referencias}

\begin{itemize}
    \item Deitel H., Deitel P. - C how to Program: With an Introduction to C++
        (2015)
    \item
        \href{https://github.com/mjkloeckner/CB100}{Repositorio de Github}
    \item
        \href{https://en.cppreference.com/w/}{C++ Reference}
    \item
        \href{https://cplusplus.com/reference/}{Standard C++ Library reference}
    \item
        \href{https://en.cppreference.com/w/cpp/container/vector_bool}{
            std::vector\textless bool\textgreater}
    \item
        \href{https://cplusplus.com/reference/fstream/ofstream/}{std::ofstream}
    \item
        \href{https://www.mingw-w64.org/}{MinGW-w64}
    \item
        \href{https://www.gnu.org/software/make/manual/html_node/index.html}{GNU
        Make Manual}
\end{itemize}

\end{document}

