\documentclass[12pt]{article}
\usepackage[spanish]{babel}
\usepackage{natbib}
\usepackage{url}
\usepackage[utf8x]{inputenc}
\usepackage{amsmath}
\usepackage{graphicx}
\usepackage{parskip}
\usepackage{fancyhdr}
\usepackage{vmargin}
\usepackage[ddmmyy]{datetime}
\usepackage{anyfontsize}
\usepackage{helvet}
\renewcommand{\familydefault}{phv}
\usepackage{xcolor}
\setmarginsrb{3 cm}{1.5 cm}{3 cm}{2.5 cm}{1 cm}{1.5 cm}{1 cm}{1.5 cm}

\usepackage{color}   % May be necessary if you want to color links
\usepackage{hyperref}
\hypersetup{
    colorlinks=true, % set true if you want colored links
    linktoc=all,     % set to all if you want sections and subsections linked
    linkcolor=blue,  % choose some color if you want links to stand out
	urlcolor=blue,
}

% \usepackage{subcaption}
\usepackage{subfig}
\usepackage[labelformat=empty]{caption}

\usepackage{fancyvrb}
\fvset{xleftmargin=\mathindent}
\usepackage{verbatimbox}

\newenvironment{fullgrayverb}
{\verbbox}
{\endverbbox\par\colorbox{gray!25}{\parbox{\textwidth}{\theverbbox}}\par}

\usepackage{tcolorbox}
% \newcommand{\code}[1]{\mbox{
%     \ttfamily
%     \tcbox[
%         on line,
%         boxsep=0pt, left=2pt, right=2pt, top=2pt, bottom=2.0pt,
%         toprule=0pt, rightrule=0pt, bottomrule=0pt, leftrule=0pt,
%         oversize=0pt, enlarge left by=0pt, enlarge right by=0pt,
%         colframe=white, colback=black!12,
%         height=.75\baselineskip
%     ]{#1}
% }}

% inline code blocks
\newcommand\mystrut{\rule[-1pt]{0pt}{.8em}}

\newtcbox{\code}{on line, boxrule=0pt, boxsep=0pt, top=2pt,
left=2pt, bottom=2pt, right=2pt, colback=gray!25, colframe=white,
fontupper={\ttfamily\mystrut}}

%%%%%%%%%%%%%%%%%%%%%%%%%%%%%%%%%%%%%%%%%%%%%%%%%%%%%%%%%%%%%%%%%%%%%%%%%%%%%%%%
% DATOS GENERALES

\title{Números Primos}          % Titulo del trabajo.
\author{Martin J. Klöckner}     % Nombre y apellido
\newcommand{\padron}{123456}    % Padrón
\newcommand{\tpnumber}{1}       % Número de trabajo práctico
\date{\today}                   % Fecha (automática)

\makeatletter
\let\thetitle\@title
\let\theauthor\@author
\let\thedate\@date
\makeatother

\pagestyle{fancy}
\fancyhf{}
\rhead{\theauthor}
\lhead{\thetitle}
\cfoot{\thepage}

%%%%%%%%%%%%%%%%%%%%%%%%%%%%%%%%%%%%%%%%%%%%%%%%%%%%%%%%%%%%%%%%%%%%%%%%%%%%%%%%
\begin{document}

%% CARATULA - NO TOCAR
\begin{titlepage}
    \includegraphics[scale = 0.75]{img/logofiuba.png}\\[1.5 cm]	    % Logo fiuba
    \centering
	\textsc{\Large CB100}\\[0.2 cm]
	\textsc{\large Algoritmos y Estructuras de Datos}\\[4 cm]
	\textcolor{cyan}{{\fontsize{40}{60}\selectfont \bfseries \thetitle}}\\[0.5cm]
	{\Large \bfseries Trabajo Práctico N$^\circ$\tpnumber}\\[5cm]
	

    \vfill
    \noindent\makebox[\linewidth]{\rule{\textwidth}{0.4pt}}\\[0.5cm]
    \begin{minipage}{.46\textwidth}
    \textbf{Autor}\\
    \theauthor
    \end{minipage}%
    \begin{minipage}{.34\textwidth}
    \textbf{Legajo}\\
    \padron
    \end{minipage}%
    \begin{minipage}{.2\textwidth}
     \begin{flushright}
        \textbf{Fecha}\\
        \thedate
    \end{flushright}
    \end{minipage}
\end{titlepage}

%%%%%%%%%%%%%%%%%%%%%%%%%%%%%%%%%%%%%%%%%%%%%%%%%%%%%%%%%%%%%%%%%%%%%%%%%%%%%%%%

{
    % \hypersetup{linkbordercolor=black}
    \hypersetup{linkcolor=black} % colorlinks=true option is used
    \tableofcontents
    \pagebreak
}

%%%%%%%%%%%%%%%%%%%%%%%%%%%%%%%%%%%%%%%%%%%%%%%%%%%%%%%%%%%%%%%%%%%%%%%%%%%%%%%%

\section{Introducción}

En el presente trabajo práctico se explica el desarrollo de una aplicación de
consola que busca números primos hasta un valor máximo determinado por el
usuario. El lenguaje utilizado para el desarrollo de la aplicación es C++ y el
algoritmo utilizado para hallar los números primos es la Criba de Eratóstenes,
aunque también se hace una mención a un método iterativo que es más intuitivo
pero menos eficiente.

\subsection{Números Primos}

Los números primos por definición son aquellos números positivos que solo son
divisibles por \code{1} y por si mismos, divisibles en términos de que el resto
de la división es cero. El \code{0} y el \code{1} son dos casos particulares en
los cuales ambos son considerados no primos, el \code{0} por razones obvias, no
está definida la división por \code{0}, y el \code{1} por que solo es divisible
por si mismo lo cual hace que solo sea divisible por un número.

El numero \code{7}, por ejemplo, es un numero primo, ya que solo es divisible
por \code{1} y por \code{7}. El numero \code{8} no es un numero primo, ya que
ademas de ser divisible por si mismo y por \code{1}, es divisible por \code{4} y
por \code{2}.

\subsection{Algoritmo Iterativo para Hallar Números Primos}

El primer algoritmo que surge casi instantáneamente es un método iterativo, el
cual comprueba numero a numero si existe algún múltiplo menor distinto de si
mismo y de \code{1}, en el caso de existir alguno, el numero queda descartado de
la lista de números primos.

\subsection{Criba de Eratóstenes}

La criba de Eratóstenes es un algoritmo más sofisticado y eficiente que el
método iterativo mencionado previamente. El algoritmo resulta conveniente para
descartar números no primos de una lista de números naturales. Teniendo una
lista de números itera numero a numero, comenzando por el primero, si se
trata de un numero primo, busca todos los múltiplos en la lista de números y
los tacha, luego procede al siguiente numero no tachado, el cual siempre
resulta un numero primo, y tacha sus múltiplos, así hasta llegar a la raíz
cuadrada del numero máximo de la lista.

\section{Desarrollo}

Para el desarrollo de la aplicación, como bien se mencionó en la introducción,
se utilizó puramente el lenguaje C++, en particular en su versión estándar C98.
Para organizar el proceso de compilación se utilizó la herramienta \code{make}.

\subsection{Implementación en C++}

Para la implementación de la aplicación en C++ se utiliza la Librería estándar
para escribir sobre archivos, así como también la librería \code{std::vector}
para representar los números a determinar si son primos.

\subsection{Reutilización de Makefile}

Ademas de la importancia de buenas practicas en el desarrollo de la aplicación,
también es importante el desarrollo de buenas practicas en el proceso de
compilación. 

El siguiente fragmento del archivo Makefile puede ser reutilizado para cualquier
proyecto, incluso aquellos con diferentes librerías

\begin{fullgrayverb}[\mbox{}]
CC := g++
CFLAGS := -Wall -Wshadow -pedantic -ansi -std=c++98 -O3
SRCS := $(wildcard *.cpp)

TARGET := primos

all: $(TARGET)

$(TARGET): $(SRCS)
    $(CC) $(CLIBS) $(CFLAGS) -o $@ $^
\end{fullgrayverb}$

\subsection{Optimizaciones del Compilador}

La mayoría de los compiladores de C++ proporcionan una opción para activar las
optimizaciones. Al tener esta opción activada el compilador intenta eliminar
código redundante para reducir el uso de memoria, tiempo de ejecución y por
consiguiente el consumo de energía, entre otras cosas.

% Teniendo un compilador que proporcione optimizaciones del código resulta
% conveniente activarla ya que reduce el tiempo de ejecución hasta en un
% \code{406\%}.

En el siguiente caso se muestra la salida del programa utilizando el máximo de
optimizaciones posible del compilador \code{-O3}, se puede apreciar en la salida
del programa el tiempo de ejecución del mismo.

\begin{fullgrayverb}
Se encontraron `5761455` números primos en `11.36` segundos
\end{fullgrayverb}

Desactivando las optimizaciones del compilador y ejecutando nuevamente la
aplicación, se obtiene el siguiente mensaje en la consola:

\begin{fullgrayverb}
Se encontraron `5761455` números primos en `46.21` segundos
\end{fullgrayverb}

Como se puede observar en el segundo caso, el tiempo de ejecución se incrementa
en un \code{406\%}.

\section{Instalación}

Para la instalación usted debe de disponer de una copia del código fuente, si
no posee una, puede obtenerla ingresando al
\href{https://github.com/mjkloeckner/CB100}{Repositorio de Github}, de allí
podrá descargar una copia del código fuente o bien puede clonar el repositorio
utilizando \code{git} con el siguiente comando:

\begin{fullgrayverb}
$ git clone https://github.com/mjkloeckner/CB100.git
\end{fullgrayverb}$

\subsection{Sistemas basados en UNIX}

Compruebe que este el mismo directorio que el archivo \code{main.cpp}

Compile el código con el programa \code{make}

\begin{fullgrayverb}
 $ make
\end{fullgrayverb}$

Luego puede ejecutar la aplicación de la siguiente manera

\begin{fullgrayverb}
 $ ./primos
\end{fullgrayverb}$

\subsection{Windows}

En el caso de Windows puede configurar
\href{https://en.wikipedia.org/wiki/Windows_Subsystem_for_Linux}{WSL},
para hacerlo puede seguir la
\href{https://learn.microsoft.com/en-us/windows/wsl/install}{guia oficial de
Microsoft}. Una vez configurado WSL siga los pasos para los sistemas basados en
UNIX

De manera análoga si usted dispone de un compilador de C++ instalado puede
compilar la aplicación directamente desde la consola, sin la necesidad de
utilizar la herramienta \code{make}, para eso ejecute el siguiente comando:

\begin{fullgrayverb}
 $ g++ -Wall -Wshadow -ansi -std=c++98 -O3 main.cpp -o primos
\end{fullgrayverb}$

Tenga en cuenta la sintaxis de su compilador ya que puede variar, el comando
anterior está previsto para \code{MinGW}

Luego de compilado la aplicación usted la puede ejecutar de la siguiente manera 

\begin{fullgrayverb}
 $ ./primos
\end{fullgrayverb}$

\section{Conclusión}

\section{Bibliografía}

\begin{itemize}
	\item \href{https://github.com/mjkloeckner/CB100}{Repositorio de Github}
\end{itemize}

\end{document}
